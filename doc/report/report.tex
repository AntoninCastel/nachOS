\documentclass{report}

\usepackage[a4paper, total={6in, 9.5in}]{geometry}
\usepackage{listings}
\usepackage{indentfirst}
\usepackage[utf8]{inputenc}
\newcommand{\myparagraph}[1]{\paragraph*{#1}\mbox{}\\}

\title{Rapport NachOS}
\author{Antonin Castel \and
		Cyprien Eymond Laritaz \and
		Marianne Garnier \and
		Robin Blanc}

\begin{document}
\maketitle

\section*{Fonctionnalités}
Notre implémentation de NachOS permet de lancer des programmes écrits en C par l'utilisateur.
Cette implémentation fournit des fonctionnalités d'entrées/sorties, de programmation avec plusieurs threads et plusieurs processus ainsi qu'un système de fichiers.

\section*{Spécifications}
\subsection*{Terminaison}
\myparagraph{Exit}

\textsc{Synopsis}\\
	\texttt{void Exit(int code)}\\
	
\textsc{Description}\\
	Permet de terminer le programme utilisateur en renvoyant le code d'erreur \texttt{code}\\
	
\textsc{Notes}\\
	Dans le cas où cette fonction est appelée par le thread "main" du processus principal, elle consiste en un appel à \texttt{Halt} précédé d'une récupération du code d'erreur afin de faire retourner ce code par le programme. Si \texttt{code} = 0, alors le programme s'est terminé sans erreur.
		
\myparagraph{Halt}

\textsc{Synopsis}\\
	\texttt{void Halt()}\\
	
\textsc{Description}\\
	Permet d'arrêter la machine.\\

\subsection*{Entrées/sorties}
\myparagraph{PutChar}

\textsc{Synopsis}\\
	\texttt{void PutChar(char c)}\\
	
\textsc{Description}\\
	Permet d'imprimer sur la sortie standard le caractère codé par le paramètre \texttt{c}.\\
	
\textsc{Notes}\\
	Lorsque le caractère passé en paramètre est \texttt{EOF}, le caractère n'est pas imprimé.




\myparagraph{PutString}

\textsc{Synopsis}\\
	\texttt{void PutString(char *string)}\\
	
\textsc{Description}\\
	Permet d'imprimer la chaîne de caractères \texttt{string} passée en paramètre.\\
	
\textsc{Notes}\\
	La longueur de la chaine ne peut être supérieure à 200 caractères.


\myparagraph{GetChar}

\textsc{Synopsis}\\
	\texttt{char GetChar()}\\
	
\textsc{Description}\\
	Permet de récupérer un caractère sur l'entrée standard.\\
	
\textsc{Valeur de retour}\\
	Un \texttt{char} correspondant au caractère lu sur l'entrée standard.\\
	
\textsc{Notes}\\
	Cette fonction est bloquante tant qu'il n'y a pas de caractère à lire sur l'entrée standard.
	



\myparagraph{GetString}

\textsc{Synopsis}\\
	\texttt{void GetString(char *string, int len)}\\
	
\textsc{Description}\\
	Permet de récupérer jusqu'à \texttt{len} caractères de l'entrée standard afin de les stocker dans la chaîne de caractères \texttt{string}.\\
	
\textsc{Notes}\\
	La lecture s'arrête si les caractères \texttt{'$\backslash$0'} ou \texttt{'$\backslash$n'} sont detectés.\\
	La taille limite des chaînes de caractères traîtées est de 200.	\\
	%Lorsque \texttt{EOF} est lu sur l'entrée standard, la lecture s'arrête même si moins de \texttt{n} caractères ont été lus.\\
	%Le nombre maximum de caractères pouvant être lus est 200.\\
	
	
	
	
\myparagraph{PutInt}

\textsc{Synopsis}\\
	\texttt{void PutInt(int n)}\\
	
\textsc{Description}\\
	Permet d'imprimer sur la sortie standard la valeur de l'entier \texttt{n} passé en paramètre.\\
	

\myparagraph{GetInt}

\textsc{Synopsis}\\
	\texttt{void GetInt(int *n)}\\
	
\textsc{Description}\\
	Permet de récupérer un entier sur l'entrée standard et le stocker dans la variable pointée par \texttt{n} passée en paramètre.\\
	
\textsc{Notes}\\
	Cette fonction est bloquante tant qu'il n'y a pas d'entier à lire sur l'entrée standard.
	

\subsection*{Multithreading}
\myparagraph{UserThreadCreate}

\textsc{Synopsis}\\
	\texttt{int UserThreadCreate(void f(void *arg), void *arg)}\\
	
\textsc{Description}\\
	Permet de créer un nouveau thread utilisateur qui va exécuter la fonction \texttt{f} avec son argument \texttt{arg}. \texttt{f} et \texttt{arg} sont donnés en paramètre de la fonction.\\
	
\textsc{Valeur de retour}\\
	Un entier qui correspond à l'identifiant du thread créé ou -1 en cas d'erreur.\\
	
\textsc{Notes}\\
	Un appel de la fonction \texttt{UserThreadExit} est nécessaire à la fin de la fonction \texttt{f} afin de terminer correctement le thread. Sans cet appel, le comportement correct du programme n'est pas garanti.\\
	Il est possible de lancer 50 threads par processus.

\myparagraph{UserThreadExit}

\textsc{Synopsis}\\
	\texttt{void UserThreadExit()}\\
	
\textsc{Description}\\
	Permet de terminer un thread.\\
	
\textsc{Notes}\\
	Cette fonction doit toujours être appelée dans un chemin d’exécution d'une fonction appelée au moyen de l'appel système \texttt{UserThreadCreate}.
	

\myparagraph{UserThreadJoin}

\textsc{Synopsis}\\
	\texttt{void UserThreadJoin(int tid)}\\
	
\textsc{Description}\\
	Permet au thread courant d'attendre la fin de l'exécution du thread dont l'identifiant de thread est \texttt{tid} passé en paramètre.\\
	
\textsc{Notes}\\
	Si le thread avec pour numéro d'identification \texttt{tid} n'est pas terminé, la fonction est bloquante jusqu'à sa terminaison. Un \texttt{tid} supérieur ou égal à 50 provoque une erreur.\\
Seuls les threads se terminant par l'appel à \texttt{UserThreadExit} peuvent être attendus avec un \texttt{UserThreadJoin}. Le comportement n'est pas garanti si cette condition n'est pas respectée.


\myparagraph{Sem\_Init}
	
	\textsc{Synopsis}\\
		\texttt{semaphore\_t Sem\_Init(int n)}\\

\textsc{Description}\\
	Permet d'initialiser un sémaphore avec \texttt{n} unités disponibles.\\
	
\textsc{Valeur de retour}\\
	Un identifiant de sémaphore (int), -1 en cas d'erreur.\\

\textsc{Notes}\\
	Les identifiants de sémaphores sont uniques dans le processus en cours. Le nombre de sémaphores est limité à 20 par processus.\\
	
\myparagraph{Sem\_P}
\textsc{Synopsis}\\	
	\texttt{int Sem\_P(semaphore\_t s)}\\

\textsc{Description}\\
	Permet de prendre une unité sur le sémaphore dont l'identifiant \texttt{s} est passé en paramètre.\\
	
\textsc{Valeur de retour}\\
0 si la fonction se déroule normalement, -1 en cas d'erreur.\\

\myparagraph{Sem\_V}
\textsc{Synopsis}\\	

\texttt{int Sem\_V(semaphore\_t s)}\\

\textsc{Description}\\
	Permet de rajouter une unité sur le sémaphore dont l'identifiant \texttt{s} est passé en paramètre.\\
	
\textsc{Valeur de retour}\\
0 si la fonction se déroule normalement, -1 en cas d'erreur.\\

\myparagraph{Sem\_GetValue}

\textsc{Synopsis}\\	
	\texttt{int Sem\_GetValue(semaphore\_t s)}\\

\textsc{Description}\\
	Permet d'obtenir le nombre d'unité du sémaphore dont l'identifiant \texttt{s} est donné en paramètre.\\
	
\textsc{Valeur de retour}\\
Le nombre d'unité du sémaphore \texttt{s}, -1 en cas d'erreur.\\

\myparagraph{Sem\_Destroy}

\textsc{Synopsis}\\	
	\texttt{int Sem\_Destroy(semaphore\_t s)}\\

\textsc{Description}\\
	Permet de libérer le sémaphore dont l'identifiant \texttt{s} est passé en paramètre.\\

\textsc{Valeur de retour}\\
0 si la fonction se déroule normalement, -1 en cas d'erreur.\\
	
\textsc{Note}\\
	Les fonctions \texttt{Sem\_P}, \texttt{Sem\_V} ou \texttt{Sem\_GetValue} renvoient une erreur après destruction de \texttt{s}.\\
\newpage
\subsection*{Mémoire virtuelle}
%%%%%%%%%%%%%%%%%%%%%%%%%%%%%
\myparagraph{ForkExec}

\textsc{Synopsis}\\	
	\texttt{pid\_t ForkExec(char* filename)}\\
	
\textsc{Description}\\
	Ouvre le fichier executable \texttt{filename}, le charge dans la mémoire et le fait tourner en parallèle du processus appelant.\\
	
\textsc{Valeur de retour}\\
	Retourne un \texttt{pid\_t} identifiant le processus lancé.\\
	
%%%%%%%%%%%%%%%%%%%%%%%%%%%%%

\myparagraph{waitpid}

\textsc{Synopsis}\\	
	\texttt{void waitpid(pid\_t pid)}\\
	
\textsc{Description}\\
	Fait une attente bloquante tant que le processus identifié par \texttt{pid} n'est pas terminé.\\
	
\textsc{Notes}\\
	L'appel système est implémenté mais non suffisamment testé, son utilisation peut ne pas produire le résultat attendu.\\
	
%%%%%%%%%%%%%%%%%%%%%%%%%%%%%

\myparagraph{Proc\_Exit}

\textsc{Synopsis}\\	
	\texttt{void Proc\_Exit()}\\
	
\textsc{Description}\\
	Fait une attente bloquante tant que les processus fils du processus appelant ne sont pas terminés, puis termine le processus.\\
	
\textsc{Notes}\\
	L'appel système est implémenté mais non suffisamment testé, son utilisation ne produit pas le résultat attendu.\\
	
	
\subsection*{Système de fichiers}
%%%%%%%%%%%%%%%%%%%%%%%%%%%%%
\myparagraph{Create}

\textsc{Synopsis}\\	
	\texttt{int Create(char *name, int size)}\\
	
\textsc{Description}\\
	Permet de créer un fichier NachOS de taille \textit{size} et de nom \textit{name} dans le répertoire courant.\\	

\textsc{Valeur de retour}\\
1 si la création du fichier est possible, 0 sinon.\\

\textsc{Note}\\
	Les répertoires NachOS sont limités à 8 entrées utilisateur.\\
%%%%%%%%%%%%%%%%%%%%%%%%%%%%%
\myparagraph{Open}

\textsc{Synopsis}\\	
	\texttt{OpenFileId Open (char *name)}\\
	
\textsc{Description}\\
	Ouvre le fichier \textit{name} du répertoire courant.\\	

\textsc{Valeur de retour}\\
	Retourne un OpenFileId permettant de manipuler le fichier si l'ouverture est possible. Cet OpenFileId vaut -1 si l'ouverture est impossible.\\	

\textsc{Note}\\
	Il existe une limite de 10 fichiers ouverts simultanément dans le système. Si cette fonction est utilisée alors que cette limite est atteinte, la valeur renvoyée sera 0.\\
%%%%%%%%%%%%%%%%%%%%%%%%%%%%%
\myparagraph{Write}

\textsc{Synopsis}\\	
	\texttt{int Write (char *buffer, int size, OpenFileId id)}\\
	
\textsc{Description}\\
	Écrit le contenu de \textit{buffer}, de taille \textit{size} à la position courant dans le fichier \textit{id}.  \\	

\textsc{Valeur de retour}\\
	Retourne le nombre de bytes effectivement écrits.\\	

\textsc{Note}\\
	Les fichiers ouverts sont communs pour tout le système, un autre programme peut donc modifier le contenu et la position courante du fichier utilisé.\\	
%%%%%%%%%%%%%%%%%%%%%%%%%%%%%
\myparagraph{Read}

\textsc{Synopsis}\\	
	\texttt{int Read (char *buffer, int size, OpenFileId id)}\\
	
\textsc{Description}\\
	Récupère un nombre \textit{size} d'octets depuis la position courante du fichier \textit{id} et les place dans \textit{buffer}.  \\	

\textsc{Valeur de retour}\\
	Retourne le nombre de bytes effectivement lus.\\
	
\textsc{Note}\\
	Les fichiers ouverts sont communs pour tout le système, un autre programme peut donc modifier le contenu et la position courante du fichier utilisé.\\	
%%%%%%%%%%%%%%%%%%%%%%%%%%%%%
\myparagraph{Close}

\textsc{Synopsis}\\	
	\texttt{void Close (OpenFileId id)}\\
	
\textsc{Description}\\
	Ferme le fichier \textit{id} pour tout les programmes utilisateurs.  \\	
	
\textsc{Note}\\
	Le fichier ne sera pas plus manipulable après cette fonction. Il est nécessaire d'utiliser en paramètres un \textbf{OpenFileId} initialement retourné par la fonction Open.\\	
	

\section*{Tests utilisateurs}
\subsection*{Entrées/sorties}
\noindent
Nous avons réalisé des programmes de tests, ils permettent de
\begin{itemize}
 \item Lire un caractère sur l'entrée standard et l'imprimer sur la sortie standard.
 \item Lire un entier sur l'entrée standard et l'imprimer sur la sortie standard.
 \item Lire une chaîne de caractères sur l'entrée standard et l'imprimer sur la sortie
 standard.
 \item Imprimer les entiers -2147483648 et 2147483647 sur la sortie standard.
\end{itemize}
Le script \texttt{nachos-step1} teste toutes les fonctionnalités implémentées dans l'étape 1, soit les appels systèmes \texttt{PutChar}, \texttt{PutInt}, \texttt{GetChar}, \texttt{GetInt}, \texttt{PutString} et \texttt{GetString}.
Les tests effectués sont des cas de
\begin{itemize}
 \item \textbf{Bonne utilisation des appels systèmes}: permet de constater que les appels systèmes, lorsqu'ils sont appelés correctement, fonctionnent. Des caractères spéciaux, des chiffres correspondant aux bornes limites, négatifs, positifs sont testés pour vérifier que l'implémentation respecte la spécification.
 \item \textbf{Dépassement du nombre de caractères maximum à lire}: permet de connaître le type de problème qui peut survenir lorsque la fonction n'est pas utilisée correctement par l'utilisateur. Cela permet de rendre plus robuste les appels systèmes.
 \item \textbf{Dépassement de capacité des entiers}: Pareil que pour le type de test précédent, nous permet de détecter le comportement lorsqu'un entier trop grand est soumis à la lecture ou à l'écriture.
\end{itemize}
Les tests effectués nous ont permis de détecter une erreur sur \texttt{GetInt}, confirmé par le test sur \texttt{PutInt}. \texttt{GetInt} peut correctement lire les entiers de -2147417856 à 2147483647 (au lieu de -2147483648 à 2147483647).

\subsection*{Multithreading}
\noindent
Pour tester la gestion et la synchronisation de plusieurs threads utilisateur, nous avons écrit plusieurs programmes de test permettant d'illustrer : 
\begin{itemize}
 \item La création de threads par le thread "main" et l'attente de leur terminaison dans le "main". Cela montre le bon fonctionnement de la création d'un thread par le "main", de sa terminaison et de son attente par "main".
 \item  La création de threads en cascade (un thread créé dans "main" crée lui même un thread, qui crée lui-même un thread, etc.).  Chaque thread est attendu par son créateur. Cela met en valeur le fait que des threads créés par l'utilisateur peuvent eux même créer d'autres threads et attendre leur terminaison. Le système ne passe pas ce test, l’exécution se bloque après la création de tous les threads créés au moment de faire les \texttt{UserThreadJoin} : nous pensons qu'un interblocage se produit à ce moment là. 
 \item La création de threads en cascade où tous les threads crées sont attendus par le "main". Cela permet de voir qu'un thread n'est pas obligé d'attendre seulement les threads qu'il a crée. Contrairement au test précédent, ce test est passé avec succès par le système.

\item La synchronisation de l'affichage de chaînes de caractères par plusieurs threads, cela nous permet de tester le bon fonctionnement des sémaphores utilisateur lorsque plusieurs threads tentent d'afficher une chaîne à l'écran : sans synchronisation, les chaînes auraient de grandes chances d'être entremêlées (\texttt{PutString} n'est pas synchrone) mais en utilisant les sémaphores, les chaînes sont affichées sans être coupées.

\item La synchronisation de threads utilisateur en utilisant un producteur-consommateur. Cela nous permet de mettre en évidence le bon fonctionnement des structures de synchronisation.
\end{itemize}
\subsection*{Système de fichiers}



\section*{Implémentation}
\subsection*{Entrées/sorties}

Pour créer les entrées sorties synchrones, nous avons utilisé les opérations asynchrones fournies par la console de base de NachOS.\\
Nous avons rendu ces opérations synchrones grâce à des Sémaphores et des Verrous.\\
Les opérations synchrones \textbf{GetString} et \textbf{PutString} sont respectivement des enchaînements de \textbf{GetChar} et \textbf{PutChar} synchrones.
Nous avons pris la décision de ne protéger que \textbf{PutChar} au niveau des accès concurrents, et pas \textbf{PutString}. Plusieurs \textbf{PutString} peuvent donc avoir lieu en même temps, dans ce cas les chaînes de caractères sont entremêlées à l'écran.\\
Les fonctions permettant de manipuler des chaînes de caractères sont capable de gérer des chaînes d'au maximum 200 caractères.\\

Nous avons également implémenté les appels système \textbf{PutInt} et \textbf{GetInt}, permettant d'écrire et de lire des entiers.\\
Pour ces fonctions, nous avons dû écrire une fonction \textbf{CopyStringToMachine}, qui est l'inverse de \textbf{CopyStringFromMachine}, elle écrit une chaîne d'une certaine taille dans la mémoire de la machine MIPS à partir d'une adresse données.\\
Cette fonction est nécessaire pour écrire l'entier lu (de 4 octets) à l'adresse passée en paramètre de GetInt.
\subsection*{Multithreading}
\myparagraph{Gestion des Threads}

\noindent
Les étapes de création d'un nouveau Thread utilisateur sont les suivantes : 
\begin{itemize}
	\item Création d'un nouvel objet Thread
	\item Création d'une structure de paramètres, avec un pointeur MIPS vers la fonction à exécuter dans ce nouveau Thread et ses arguments.
	\item Fork de ce nouveau Thread, pour qu'il exécute la fonction \texttt{StartUserThread}, avec comme paramètre la structure créée auparavant.
	\item La fonction \texttt{StartUserThread} écrit les registres PC et nextPC pour que la prochaine instruction soit la première instruction de la fonction passée en paramètre, et fixe aussi le SP du nouveau Thread.
	\item On invoque ensuite \texttt{machine$\rightarrow$run()} pour lancer le nouveau Thread
\end{itemize}

Nous avons donné à chaque nouveau Thread utilisateur un identifiant unique et jamais réutilisé. Le nombre de threads par processus qui peuvent être créés est fixé à 50.\\

La pile du processus est découpée en bloc de 3 pages qui sont destinés à accueillir les piles de chaque thread utilisateur. Les séparations de ces blocs étant à des adresses précises et fixes, il faut savoir à quelle adresse doit commencer le premier bloc. Pour cela, nous avons décidé qu'après la création du premier thread utilisateur que cette borne soit placée 3 pages après le SP du thread "main" du processus. Pour gérer les allocations de blocs mémoire, on utilise une bitmap : un bit à 1 à l'indice \textit{i} indique que le bloc \textit{i} est alloué pour un thread utilisateur (et inversement pour un bit à 0). Le numéro de bloc alloué pour un thread est sauvegardé dans le thread. De cette manière, dès qu'un thread se termine il libère son bloc qui peut alors être ré-alloué par un autre thread. \\ 
   
Le thread "main" attend la terminaison des threads qu'il a créé sans que l'utilisateur n'ait besoin de le demander (il peut tout de même le faire). En effet, lorsque le thread "main" du processus se termine, l'espace d'adressage du processus est détruit : si des threads ne sont pas terminés à ce moment là et qu'ils tentent d'accéder à leur mémoire, cela produira un accès invalide à la mémoire et une erreur de segmentation fatale au système.\\
%Cependant, la terminaison des threads créés par un thread X hors thread "main" n'est pas attendue automatiquement.  Lors de la terminaison de X s'ils n'ont pas été attendus (l'espace d'adressage du processus n'est pas détruit dans ce cas là). 
Un thread X hors thread "main" qui créé d'autres threads n'attend pas automatiquement leurs terminaisons, cela s'explique par le fait que l'espace d'adressage du processus n'est pas détruit. C'est donc à l'utilisateur d'utiliser \texttt{UserThreadJoin} avec les \textit{tid} correspondants pour s'assurer de la bonne terminaison de ces threads. Un thread peut attendre n'importe quel autre thread crée par l'utilisateur (pas forcément les threads qu'il a lui même crée) et deux threads peuvent attendre la terminaison d'un même thread. Il faut remarquer que la terminaison d'un thread n'est "captée" par un \texttt{UserThreadJoin} que si le thread attendu à utilisé un appel à \texttt{UserThreadExit}.
 

\myparagraph{Synchronisation des Threads}\\
Pour synchroniser l'utilisation de plusieurs threads, le système permet à l'utilisateur de créer des sémaphores qu'il pourra contrôler avec les appels système \texttt{Sem\_Init},\texttt{Sem\_P}, \texttt{Sem\_V}, \texttt{Sem\_GetValue} et \texttt{Sem\_Destroy}. Pour créer ces appels système, nous avons créé au niveau noyau un tableau où sont stockés les sémaphores noyau destinés à être contrôlés au niveau utilisateur. Comme pour les threads utilisateurs, les identifiants de sémaphores utilisateur sont uniques : le nombre de sémaphores utilisateur par processus est fixé à 20. Lors d'un appel à \texttt{Sem\_Init}, on renvoie un identifiant de sémaphore associé un sémaphore noyau du tableau. \\

Pour savoir quels sont les sémaphores en cours d'utilisation, on utilise une bitmap où le numéro de chaque bit correspond à un indice dans le tableau.Le bit est à 1 si le sémaphore est initialisé au niveau utilisateur, 0 sinon. Cela permet de vérifier à chaque appel des fonctions \texttt{Sem\_P}, \texttt{Sem\_V}, \texttt{Sem\_GetValue} et \texttt{Sem\_Destroy} que le sémaphore que l'utilisateur souhaite manipuler a bien été initialisé et n'a pas été détruit. Si un mauvais accès est effectué (le bit correspondant est à 0), alors ces appels systèmes retournent -1 et protègent le système.
 
\subsection*{Mémoire virtuelle}
Les espaces d'adressages utilisent un \texttt{FrameProvider} qui leur alloue des pages physiques de la mémoire. De ce fait, on peut avoir plusieurs processus qui partagent la mémoire physique de la machine.\\
Chaque processus possède:
\begin{itemize}
	\item Un pid qui identifie le processus,
	\item Un sémaphore qui permet d'attendre la fin de ce processus,
	\item Une map qui indexe les pids de ses fils à leurs sémaphores.
\end{itemize}
Les conséquences de cette organisation sont que les pids sont réservés aux processus pères. Les pids étant locaux à chaque processus, un processus père n'a pas accès aux pids des processus fils de ses fils.\\
 %un processus ne peut pas contrôler les processus fils de ses fils, car les pids auquel il a accès lui sont locaux.
Les pids commencent à zéro pour tous les processus. Le nombre de processus est limité par la taille de la mémoire physique.\\

Malgré le fait que l'implémentation n'est pas fonctionnelle, le choix que nous avons fait pour la terminaison des processus est le suivant.
Lorsqu'un processus se termine, il attend la fin de tous ses fils par le biais des sémaphores auxquels il a accès.
De la même manière, ses fils doivent attendre la fin de leurs processus fils. Ainsi il n'y a pas de processus orphelin.

\subsection*{Système de fichiers}
\myparagraph{Arborescence de répertoires}
	
Pour créer une arborescence de fichiers, nous avons ajouté aux \textit{i}-nodes des fichiers plusieurs champs.
\begin{itemize}
\item \textbf{IsDirectory} permettant de savoir si le fichier correspondant à l'\textit{i}-node est un répertoire.
\item \textbf{DirectorySector} permettant de savoir sur quel secteur du disque se trouve cette \textit{i}-node.
\item \textbf{ParentDirectorySector} permettant de savoir sur quel secteur du disque trouver l'\textit{i}-node du fichier correspondant au répertoire parent.
\end{itemize}
Quand les fichiers ne représentent pas un répertoire, ces champs sont laissés vide. \\
Le répertoire Root contient l'information sur le secteur contenant l'\textit{i}-node du fichier représentant le répertoire courant, on peut donc y acceder via le répertoire \textbf{Root}, qu'on sait tout le temps où trouver car l'\textit{i}-node correspondante est toujours sur le secteur 1.	\\	
L'opération \textbf{cd} est donc simplement un changement du numéro de "secteur courant" dans le fichier du répertoire \textbf{Root}.  \\
L'opération \textbf{mkdir} créé donc un nouveau répertoire dans le répertoire courant, qui contient deux fichiers, "." et "..", qui sont respectivement le répertoire courant et le répertoire parent.\\	
	
	Ces répertoires spéciaux nous permettrons donc de remonter l'arborescence jusqu'au répertoire racine "/". Il est aussi possible de se déplacer directement dans ce répertoire.
	
	Le répertoire courant est commun à tout le système, chaque processus ou thread utilisant la fonction \texttt{Open} simultanément prendront donc en considération les fichiers présent dans le même répertoire.//
De plus, la notion de chemin n'existe pas, tout les noms de fichiers sont des noms relatifs au répertoire courant.\\

Les opérations \textbf{cd} et \textbf{mkdir} ne sont pas des appels système mais des options d'exécution de NachOS. Il faut leur donner en argument le nom du répertoire voulu.\\

Il est à noter que les noms ".", "..", "/" ainsi que la chaine de caractères vide ne sont pas des noms de fichier (ou répertoires) valides.
\myparagraph{Table système des fichiers ouverts}\\	
Pour créer la table système des fichiers ouverts, nous avons ajouté plusieurs champs à la classe \textbf{filesys} :
\begin{itemize}
\item Une table de noms, contenant les noms des fichiers ouverts.
\item Une Bitmap, permettant de savoir quelles entrées de la table sont disponibles

\item Une table d'OpenFile*, permettant d’accéder à un fichier ouvert.
\end{itemize}
L'utilisateur lui, ne manipule que les indices de ces tableaux.\\
Cette table permet de faire l'interface entre le programme utilisateur et le système de fichiers.\\

Par exemple, si un programme utilisateur veut faire un \textbf{Open} sur un fichier qui n'est pas ouvert, alors on va chercher dans la Bitmap un indice libre, mettre le nom du fichier à cet indice dans la table des noms, l'ouvrir, et mettre le OpenFile* correspondant à cet indice dans la table d'OpenFile.
La valeur renvoyée à l'utilisateur sera l'indice.\\

En revanche, si le fichier apparaît déjà dans la table des noms, alors l'indice où ce fichier à été trouvé est renvoyé à l'utilisateur, car ça signifie qu'il est déjà ouvert.\\

Comme la table est globale, il se peut qu'il y ait des conflits entre deux processus qui voudraient accéder au même fichier. Si l'un d'entre eux ferme le fichier, le second n'y aura plus accès.\\
\section*{Organisation du travail}
Lors de la première moitié sur projet, le travail s'est organisé le plus souvent avec deux machines. Ces deux machines permettent à tout le groupe d'avancer dans la programmation d'une fonctionnalité et de faire de la recherche d'autre part (prototype exact, fichiers à inclure etc...). La machine où se fait le développement et le débuggage est souvent branchée à un second écran plus grand afin que tout le groupe puisse voir clairement le code ou les étapes de débuggage.\\
Le travail a été collectif et non systématiquement réparti fonctionnalités par fonctionnalités à certains membres du groupe. Il est arrivé qu'un membre du groupe s'approprie une tâche, mais les autres membres lui ont apporté de l'aide, que ça soit directement en debuggant le code, ou par le biais d'idées et de discussion à propos de la fonctionnalité. C'est le cas des fonctionnalités plus petites, comme certains appels systèmes d'entrées/sorties.\\

Lors de la seconde moitié du projet, le travail a été partitionnés en plusieurs points; la mémoire virtuelle, le système de fichier et les tests et corrections.\\


Le groupe a alterné les phases de discussions/débats dans le but de dégager les différents problèmes ou implications de chaque fonctionnalités à développer. Cette étape a permis d'orienter la réflexion du groupe dans la même direction ainsi que d'harmoniser les connaissances de chacun et de lever des doutes et clarifier des concepts du cours ou de NachOS que tout le monde n'a pas forcément assimilé de la même façon.\\

\end{document}